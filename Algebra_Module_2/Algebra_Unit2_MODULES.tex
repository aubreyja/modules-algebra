%%%%% META DATA %%%%%%%%%%%%
\newcommand\Course{Algebra for Secondary Mathematics Teaching} % e.g., Algebra, Geometry, Modeling, Statistics
\newcommand\Location{MODULE($\textnormal{S}^2$)} % affiliation and course
\newcommand\Term{Spring 2018} % term taught

 \newcommand\MODULES{$\textnormal{MODULE}(\textnormal{S}^2)$}
 \title{{\normalsize{Mathematics Of Doing, Understand, Learning, and Educating Secondary Schools} }\\  $\;$ \\ \MODULES: \\  \Course}
 \author{Adapted for \Location} 
 \date{Version \Term} 
 
 % Input a course graphic or leave the { } empty:
\newcommand\coursegraphic{DoubleRectangles.png} 
 
 %%%%%%%DOCUMENT FORMATTING %%%%%%%%%%
\documentclass[11pt]{article}
\linespread{1.03}% 6 lpi http://tex.stackexchange.com/questions/23824/6-lines-in-one-inch
\parskip6pt
\usepackage{amsmath, amsthm, amsfonts, amssymb, mathpazo, url, graphicx, stackrel, mdwlist, enumitem, mdframed, ifthen}
	% usual suspects, palatino, hyperlink capability, PDF graphics, symbol stacking, list customizations, boxes, ifthenelse macros
\usepackage[top=1in,bottom=1in,left=1in,right=1in]{geometry} % 8.5" x 11" pages with 1 inch margins
\usepackage[pdftex, bookmarks, colorlinks, breaklinks]{hyperref} % prettier hyperlinks
\usepackage[usenames,dvipsnames,svgnames,table]{xcolor} % defines colors for text and tikz graphics
\definecolor{darkred}{rgb}{0.8,0.1,0.2} % for hyperlinks
\definecolor{darkblue}{rgb}{0.2,0.1,0.7} % for hyperlinks
\hypersetup{linkcolor=darkred,citecolor=blue,filecolor=dullmagenta,urlcolor=darkblue} % colors for links
\usepackage[none]{hyphenat} % prettier hyphenating
\raggedright \parskip4pt  \parindent0pt 
\usepackage{array} % tables with paragraphs of set widths
\renewcommand{\arraystretch}{1.3} % makes tables more legible
\newcolumntype{L}[1]{>{\raggedright\let\newline\\\arraybackslash\hspace{0pt}}p{#1}}
\newcolumntype{C}[1]{>{\centering\let\newline\\\arraybackslash\hspace{0pt}}p{#1}}
\newcolumntype{R}[1]{>{\raggedleft\let\newline\\\arraybackslash\hspace{0pt}}p{#1}}
\usepackage{rotating} % rotating figures and tables, provides sidewaystable and sidewaysfigure
\usepackage{lipsum} % text testing


%%%%%%% DOCUMENT MANAGEMENT %%%%%%%%%%%%
% To do notes and commenting 
\usepackage{comment}

% View instructor notes 
\newif\ifinstructor 
 \instructortrue  % view as instructor 
% \instructorfalse  % view as student
  
%%%%%%% SECTION FORMATTING %%%%%%%%%%%%
% sections
\usepackage{titlesec}
\titleformat{\subsection}[block]{\Large \bfseries \filcenter}{}{0em}{}
\titleformat{\subsubsection}[block]{\large \scshape\filcenter}{}{0em}{}
\newcommand{\handout}{\subsubsection}
\newcommand\header[1]{\vspace*{4pt}\par {\large {\bf #1}}\par}
\newcommand\about{\textasciitilde}

% itemize - second layer is an open circle instead of dash
\def\labelitemii{$\circ$}

% table colors - mostly for fun
\definecolor{yellow}{RGB}{255, 255, 0}
\definecolor{red}{RGB}{226, 30, 60}
\definecolor{orange}{RGB}{255, 159, 12}
\definecolor{green}{RGB}{16, 168, 112}
\definecolor{blue}{RGB}{1,200,255}
\definecolor{periwinkle}{RGB}{200,200,255}
\definecolor{lightteal}{RGB}{200,250,250}
\definecolor{purple}{RGB}{113, 1, 232}
%\definecolor{pink}{RGB}{255,70,192}
\definecolor{pink}{RGB}{232,1,193}
\definecolor{gray}{RGB}{100, 100, 100}

% instructor notes
\ifinstructor 
\newenvironment{bignote}[1][Instructor note]% default note is an "Instructor Note"
	{\begin{mdframed}\raggedright{\bf #1.~}}
	{\end{mdframed}}  
\else \excludecomment{bignote}
\fi

\ifinstructor
\newcommand\smallnote[1]
	{\begin{mdframed}\raggedright  {\bf Instructor note.} {#1} \end{mdframed}}
\else \newcommand\smallnote[1]{}
\fi

\ifinstructor  \usepackage{todonotes} 
\else \usepackage[disable]{todonotes}
\fi

% in-class task
\newenvironment{task}
	{\begin{mdframed}[linecolor=lightgray, linewidth=3pt]\raggedright}
	{\end{mdframed}}

%%% GRAPHICS / TIKZ %%%%%%%%%%%
\graphicspath{{Images/}}

\usepackage{tikz}
\usepackage{tkz-euclide} % tikz package for Euclidean geometry
\usepackage{siunitx} % typesetting quantities
\usepackage{pgfplots} %\pgfplotsset{compat=1.13} % plotting graphs
\usetikzlibrary{calc} % calculations within tikz
\usetkzobj{all} % needed for tkz-euclide package
 
%%%%% MATH NOTATION %%%%%%%%

\newcommand\tn{\textnormal}

% systems
\newcommand{\R}{\mathbb{R}}
\newcommand{\C}{\mathbb{C}}
\newcommand{\Q}{\mathbb{Q}}
\newcommand{\N}{\mathbb{N}}
\newcommand{\Z}{\mathbb{Z}}

% notation tweaking
\renewcommand\phi\varphi  % normal \phi looks too much like the empty set.
\renewcommand\subset\subseteq 
\renewcommand\supset\supseteq  % to be careful about strict subsets and nonstrict subsets
\newcommand\st{:}

% divisibility
\newcommand\divides{\;|\;}
\newcommand\notdivides{\hspace*{-2pt}\not |\;}

% trig and geometry
\newcommand\degrees{^\circ}

%%%%%%%%%% THEOREMS AND RELATED STRUCTURES %%%%%
\renewcommand\emph[1]{\underline{\bf{#1}}} % terminology

\newtheorem{theorem}{Theorem}[section]
\newtheorem{proposition}[theorem]{Proposition}
\newtheorem{lemma}[theorem]{Lemma}
\newtheorem{corollary}[theorem]{Corollary}
\newtheorem{claim}{Claim}

\theoremstyle{definition}
\newtheorem{definition}[theorem]{Definition}
\newtheorem{example}[theorem]{Example}
\newtheorem{problem}[theorem]{Problem}
\newtheorem{conjecture}[theorem]{Conjecture}
\newtheorem{question}[theorem]{Question}
\newtheorem{remark}[theorem]{Remark}
\newtheorem{case}{Case}

\newtheorem*{theorem*}{Theorem}
\newtheorem*{example*}{Example}
\newtheorem*{question*}{Question}
\newtheorem*{claim*}{Claim}
\newtheorem*{definition*}{Definition}

\newenvironment{solution}{{\it Solution.} }{\hfill {\color{lightgray}$\blacksquare$}}

\newcommand\qedpart[1]{ \hfill \framebox(6,6){\tiny #1}}
\renewcommand\qed{\hfill \framebox(6,6){}}
%%%%%%%%%%%%%%%%%%%%%%%%%%%%%%%% 	
%%%%%%%% DOCUMENT BEGINS %%%%%%%%%%%%
%%%%%%%%%%%%%%%%%%%%%%%%%%%%%%%% 

\begin{document}

%%%%%% COVER PAGE %%%%%%%%%%%%%%% 
\pagenumbering{gobble} % no page number
\maketitle
\ifthenelse{\equal{\coursegraphic}{}} % insert course graphic if one exists
	{}
	{\begin{center}\includegraphics[width=3in]{\coursegraphic}\end{center}}
	
\vfill 
% copyleft
\begin{center} \includegraphics[width=1in]{by-nc-sa.png} \end{center}
\footnotesize{ This work is licensed under a Creative Commons Attribution-ShareAlike 3.0 Unported License. }

 % acknowledgments 
\footnotesize{
The Mathematics Of Doing, Understand, Learning, and Educating Secondary Schools (\MODULES) project is partially supported by funding from a collaborative grant of the National Science Foundation under Grant Nos.~DUE-1726707,1726804, 1726252, 1726723, 1726744, and 1726098.  Any opinions, findings, and conclusions or recommendations expressed in this material are those of the authors and do not necessarily reflect the views of the National Science Foundation.}
\newpage
%%%%%% TABLE OF CONTENTS %%%%%%%%%%%%%%% 	
\thispagestyle{plain} \pagenumbering{roman}  
\listoftodos
\tableofcontents
\newpage \pagenumbering{arabic}
%%%%%%%%%%%%%%%%%%%%%%%%%%%%%%%% 
%%%%%%%%%%%%%%%%%%%%%%%%%%%%%%%% 	
%%%%%% PART 1 %%%%%%%%%%%%%%%%%%%%%	
%%%%%%%%%%%%%%%%%%%%%%%%%%%%%%%% 
%%%%%%%%%%%%%%%%%%%%%%%%%%%%%%%% 	
\newpage 

\part{Introduction to Complex Numbers}

\section{Basics of Complex Numbers} \label{section: basics of complex numbers}

\subsection{Defining Complex Numbers}

First, a bit of review. The set of natural numbers, $\mathbb{N}$, consists of all of the counting numbers:
\[ \mathbb{N} = \{ 0, 1, 2, 3, \dots \}.\]
The set of integers, $\mathbb{Z}$, consists of the natural numbers and their additive inverses:
\[ \mathbb{Z} = \{ \dots,-3,-2,-1,0, 1, 2, 3, \dots \}.\]
The rational numbers are those numbers that can be written as a fraction:
\[ \mathbb{Q} = \{ \frac{a}{b} \mid a, b\in \mathbb{Z},\, b \neq 0 \}.\]
Some numbers cannot be written as a ratio of two integers, and we call those numbers irrational. For example, $\sqrt{2}$ is not a rational number; it is
irrational. But, finally, we have the set of real numbers, $\mathbb{R}$, which consists of all the rational numbers together with all of the irrational numbers. 
So, we have the following containments:
\[ \mathbb{N} \subseteq \mathbb{Z} \subseteq \mathbb{Q} \subseteq \mathbb{R}.\]
We often model the real numbers on a line, called the real number line. 
\begin{center}
  INSERT IMAGE
\end{center}
Real numbers can added, subtracted, multiplied and divided. As suggested by the number line model, there is also an ordering of the real 
numbers with the following properties:
\begin{itemize}
  \item For any real numbers $a$ and $b$, if $a\leq b$ and $b\leq a$, then $a=b$. (Antisymmetry)
  \item For any real numbers $a$, $b$, and $c$, if $a \leq b$ and $b\leq c$, then $a\leq c$. (Transitivity)
  \item For any real numbers $a$ and $b$, $a\leq b$ or $b\leq a$. (Totality)
\end{itemize}

These properties fit our intuition for how objects on a number line should behave, and for that reason we call an ordering with these properties a 
\emph{linear order}. So, we say the real numbers are ``linearly ordered.'' The real numbers also have a very important property called ``completeness,'' 
which allows us to conclude that there are no ``gaps'' in the real numbers. Although the real numbers are complete in this sense, it turns out that there 
are some problems which can't be solved with real numbers.

\subsection{Inquiry: Sums and Products}
\begin{task}
  Find two numbers whose sum is 10 and whose product is 16.

  Find two numbers whose sum is 10 and whose product is 25.

  Find two numbers whose sum is 10 and whose product is 40.
\end{task}

\smallnote{
Distribute handout with this question. As teachers work on it, circulate and listen to the questions and comments they make. They may say and do things that will lead into a discussion on clarifying the question, precision, and also what it means to have less or more satisfying answers to a question.
}

In the last task in the inquiry above, the equation
\[ x^2 - 10x + 40 = 0\]
is found to have no real solutions. So are there two numbers which satisfy the requirements?  Well, there are no \emph{real} numbers that work, but
we can find two \emph{complex numbers}: $5 + i\sqrt{15}$ and $5-i\sqrt{15}$. In the sections that follow we will introduce complex numbers and study their 
properties.

\subsubsection{Ordered Pairs}

You may know that complex numbers have the form $a + bi$ where $a$ and $b$ are real numbers and $i = \sqrt{-1}$. And, you may also know that we can model 
complex numbers in the plane. We will begin our study of complex numbers by viewing them as points or vectors in the plane. Thus, we can say that a 
\emph{complex} number $z$ is an ordered pair
\[ z = (a,b)\]
of real numbers $a$ and $b$. This representation allows us to very naturally graph complex numbers in the plane:

\begin{center}
  INSERT IMAGE
\end{center}

When we graph complex numbers in the plane, we call that plane the ``complex plane.''

Also, in this representation, it is natural to identify a complex number of the form $(a,0)$ with the real number $a$. Thus we identify the complex number
$(3,0)$ with the real number 3 and we identify the complex number $\left( \frac{1}{3},0 \right)$ with the real number $\frac{1}{3}$. 

Thus, if we denote the set of complex numbers by the symbol $\mathbb{C}$, then we have
\[ \mathbb{N} \subseteq \mathbb{Z} \subseteq \mathbb{Q} \subseteq \mathbb{R} \subseteq \mathbb{C}.\]
In this way, the complex numbers contain the real numbers as a subset. In the complex plane we say that the set of real numbers are found along the ``real axis.''

Complex numbers of the form $(0,b)$ are called \emph{pure imaginary numbers}, or just \emph{imaginary numbers}. In the complex plane, the set of imaginary numbers
are found along the ``imaginary axis.''

\begin{center}
  INSERT IMAGE
\end{center}

If we are given a complex number $z=(a,b)$ we call $a$ the \emph{real part} of $z$ and we call $b$ the \emph{imaginary part} of $z$. The 
notation for this is 
\[ \text{Re }z = a \text{ and } \text{Im }=b.\]

Two complex numbers are said to be equal when they have the same real parts and the same imaginary parts. That is (just like ordered pairs), we say
\[ (a,b) = (c,d) \text{ if and only if } a=c \text{ and } b=d. \]

Now, notice that
\[ z = (a,b) = a \cdot (1,0) + b \cdot (0,1). \]
We already know that we can identify the complex number $(1,0)$ with the real number 1. We will also \emph{let $i$ denote the pure imaginary number $(0,1)$.}
Thus, we can write
\begin{align*}
 z &= (a,b)\\ 
   &= a \cdot (1,0) + b \cdot (0,1)\\
   &= a\cdot 1 + b\cdot i\\
   &= a + bi
\end{align*}

Many students are more familiar with the form $z = a + bi$ for the complex number $(a,b)$. Both representations are useful. We will usually
use the form $z = a+bi$, but we will also often use the form $z=(a,b)$. The notation $a+bi$ is very flexible, but there are some conventions. For example,
$5+3i$, $5+i3$, $3i+5$ and $i3+5$ are all equal, but we usually write it as $5+3i$. Instead of writing something like $8+1i$, we would usually 
just write $8+i$. And, rather than $3+0i$ or $0+0i$ we typically just write $3$ or $0$, respectively.

\begin{task}
  \begin{itemize}
    \item Let $z=a+bi$ where $a,b\in \mathbb{R}$. There are two triangles that contain $z$ as a vertex and whose legs are parallel or perpendicular to the real and imaginary
  axes. What are the coordinates of the other vertices of these triangles in terms of complex numbers?
  \item In the above question, what are the lengths of the legs and hypotenuses of the right triangles?
  \end{itemize}
\end{task}

Given a complex number $z=(a,b)$ the \emph{complex conjugate} of $z$ is the complex number $z=(a,-b)$.

\begin{task}
  \begin{itemize}
    \item What is the complex conjugate of $z=3-4i$? Graph $z$ and $\overline{z}$ on the same set of axes. 
    \item What is the complex conjugate of $z=(0,1)$? Graph $z$ and $\overline{z}$ on the same set of axes.
    \item What transformation of the plane maps $z$ to $\overline{z}$?
  \end{itemize}
\end{task}

\subsection{Addition and Subtraction of Complex Numbers}

Complex numbers are identified with points in the plane for some very good reasons. One of those reasons is that addition works as expected: we
add them \emph{component wise}. That is, given complex numbers $z_1=(a,b)$ and $z_2 = (c,d)$ we add them as follows:
\begin{align*}
  z_1 + z_2 &= (a,b)+(c,d)\\
            &= (a+c, b+d)
\end{align*}
If we had represented these complex numbers as $z_1=a+bi$ and $z_2=c+di$ then addition would work as follows
\begin{align*}
  z_1 + z_2 &= (a+bi)+(c+di)\\
            &= (a+c) + (b+d)i
\end{align*}

\begin{task}
  In any number system, we say a number $\mathbf{n}$ is an \emph{additive identity} if $x + \mathbf{n} = \mathbf{n} + x = x$ for any number $x$ in that 
  number system.
  \begin{itemize}
    \item What is the additive identity of the natural numbers? What about the integers, rationals, and reals?
    \item What is the additive identity in the set of complex numbers $\mathbb{C}$? Write it in two ways.
  \end{itemize}
  Suppose $x$ is a number of any type. It can be an integer, rational, real or complex number. We define the additive inverse of $x$ to be the number $y$ so
  that $x+y = \textbf{0}$. 
\begin{itemize}
  \item Given a complex number $z = a+bi$, what is its additive inverse? 
  \item How would you write $z$ and its additive inverse using ordered pairs?
  \item How would you define subtraction of complex numbers? Write the definition using both the $a+bi$ representation and the ordered pair representation.
  \item Given a complex number $z$, what is $z+\overline{z}$? What is $z-\overline{z}$? Write your answers in terms of $\text{Re }z$ and $\text{Im }z$.
\end{itemize}
\end{task}

Finally we define the \emph{modulus} of a complex number $z=a+bi$ to be $|z| = \sqrt{a^2+b^2}$. The modulus goes by many names. You will also see the modules of $z$
referred to as the \emph{magnitude of $z$}, the \emph{norm of $z$}, the \emph{length of $z$}, and sometimes the \emph{absolute value of $z$}.

\begin{task}
  \begin{itemize}
    \item Use the Pythagorean theorem to explain why it makes sense that the modulus of $z$ is also called the magnitude (or norm, or length) of $z$.
    \item What is the relationship between the concept of the absolute value of a real number and the concept of the modulus of a complex number?
  \end{itemize}
\end{task}


\subsection{Inquiry: The Mathematics of Turns}

TODO: Still editing this.

\subsubsection{The Mathematics of Half Turns}

For the tasks below you should be in groups of 3-4 people, and you will need blank paper.

\begin{task}
\begin{itemize}
  \item Stand up and face forward. Describe what you see to the other members of your team. Do a half-turn counterclockwise. Again, describe what you see to your team members. 
    Do another half-turn counterclockwise. Describe what you see. Do a full turn. Describe what you see. 
  \item Draw a picture that illustrates what you just did.
  \item Describe what just happened. In your description you should (at least) answer the following questions: What did you do? What did you see? Did everyone see the same thing? How did you decide what ``forward'' was? 
    What was interesting? What was boring?
  \item Identify the important features from the description that you wrote above. Share your features with your group. If someone from your team identifies a feature that you didn’t, and you also 
    think it is important, include it in your list.
  \item Create an icon that represents each of the features you listed above. Remember that, though they may both be considered graphics, there is an important difference between icons and symbols. (???)
  \item Use your icons to write an iconic sentence that describes the experience you had at the start of this activity.
\end{itemize}
\end{task}

\subsubsection{Thinking About the Mathematics of Half Turns}

The table shown below is a special type of multiplication table. Note that the symbol in the top left cell of the table is meant to represent the verb ``followed by.''
\begin{center}
  \begin{tabular}{|c|c|c|}\hline
    $\ast$      & \textbf{HT} & \textbf{FT} \\ \hline
    \textbf{HT} &             & \\ \hline
    \textbf{FT} &             & \\  \hline
  \end{tabular}
\end{center}
\begin{task}
\begin{itemize}
  \item Do you have any ideas about what the symbols in the table are meant to represent? Write down your ideas.
  \item Do you have any ideas about how to complete the table? Write down your ideas.
  \item Share your ideas with the other members in your team. Make a conjecture about how to complete the table and be prepared to share your conjecture with the class. Include the rationale behind your conjecture.
  \item Complete the table.
  \item Do you notice anything interesting about the completed table? If no, then what is it that makes the completed table so dull?
  \item Think of the other mathematical concepts that you have learned about in your career as a student. Can you create another table that combines those concepts in the 
    same way as the table above? If so, draw and explain it. Please be ready to share it with the class.
\end{itemize}
\end{task}

\subsubsection{The Mathematics of Quarter Turns}

\begin{task}
\begin{itemize}
  \item Stand up and face forward. Describe what you see to the other members of your team. Do a quarter-turn counterclockwise. Again, describe what you see to your team members. Do another quarter-turn 
    counterclockwise. Describe what you see. 
  \item Draw a picture that illustrates what you just did.
  \item Write a description of what just happened. In your description you should (at least) answer the following questions: What 
    did you do? What did you see? Was there anything about this experience that was the same as your experience last time? What was interesting? What was boring?
  \item Identify the important features from the description that you wrote above. Share your features with your group. If someone from your team identifies a feature 
    that you didn’t, and you also think it is important, include it in your list.
  \item Create an icon that represents each of the features you listed above. Remember that, though they may both be considered graphics, there is an important difference between icons and symbols. (???)
  \item Use your icons to write an iconic sentence that describes the experience you had at the start of this activity.
\end{itemize}
\end{task}

\subsubsection{Thinking Hard About Quarter Turns}

The multiplication table shown to the right is much like the one you saw when working with half-turns.  The symbol in the top left corner still represents the feature ``followed by.'' You can 
probably guess what the symbol ``QT'' represents\dots
\begin{center}
  \begin{tabular}{|c|c|c|c|}\hline
    $\ast$      & \textbf{HT} & \textbf{FT} & \textbf{QT} \\ \hline
    \textbf{HT} &             &  & \\ \hline
    \textbf{FT} &             &  & \\  \hline
    \textbf{QT} &             &  & \\  \hline
  \end{tabular}
\end{center}
\begin{task}
\begin{itemize}
  \item Make a conjecture about how to complete the table and be prepared to share your conjecture with the class. Include the rationale behind your conjecture.
  \item Do you notice anything interesting about the completed table? If no, then tell us what’s boring about the table.
  \item How is this completed table the same and/or different from the table you completed about half-turns?
  \item Think of the other mathematical concepts that you have learned about in your career as a student. Can you create another table that combines those concepts in the same way as the table above? If so, 
    draw and explain it below. Please share it with the class.
  \item Consider the two tables below. They are the two tables that we discussed a few days ago while thinking about the mathematics of turns:
    \begin{center}
      INSERT TABLES
    \end{center}
  \item Now consider the two tables below that summarize the nature of quarter-turns. First, complete the table on the left in the same way that you did at the beginning of this work. Can 
    you complete the table on the right so that it can be ``substituted for'' the table on the left? As a hint, think about the two tables on the previous page.
    \begin{center}
      INSERT TABLES
    \end{center}
\end{itemize}
\end{task}

\subsection{Multiplication of Complex Numbers}

Before we discuss multiplication of complex numbers in full generality, let us discuss multiplication by $i$. As you may know, we define
\[ \sqrt{-1} = i. \]
So the symbol $i$ is here used as shorthand for $\sqrt{-1}$. But as you have seen we can also represent $i$ as an ordered pair $(0,1)$.

TODO: RELATE TO ABOVE INQUIRY

Multiplication of two complex numbers is defined as follows:
\[ (a+bi)(c+di) = (ac-bd) + (ad+bc)i.\]

\part{Representations of Complex Numbers}

\part{Roots and polynomials}



\end{document}



