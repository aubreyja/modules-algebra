%%%%% META DATA %%%%%%%%%%%%
\newcommand\Course{Algebra for Secondary Mathematics Teaching} % e.g., Algebra, Geometry, Modeling, Statistics
\newcommand\Location{MODULE($\textnormal{S}^2$)} % affiliation and course
\newcommand\Term{Spring 2018} % term taught

 \newcommand\MODULES{$\textnormal{MODULE}(\textnormal{S}^2)$}
 \title{{\normalsize{Mathematics Of Doing, Understand, Learning, and Educating Secondary Schools} }\\  $\;$ \\ \MODULES: \\  \Course}
 \author{Adapted for \Location} 
 \date{Version \Term} 
 
 % Input a course graphic or leave the { } empty:
\newcommand\coursegraphic{DoubleRectangles.png} 
 
 %%%%%%%DOCUMENT FORMATTING %%%%%%%%%%
\documentclass[11pt]{article}
\linespread{1.03}% 6 lpi http://tex.stackexchange.com/questions/23824/6-lines-in-one-inch
\parskip6pt
\usepackage{amsmath, amsthm, amsfonts, amssymb, mathpazo, url, graphicx, stackrel, mdwlist, enumitem, mdframed, ifthen}
	% usual suspects, palatino, hyperlink capability, PDF graphics, symbol stacking, list customizations, boxes, ifthenelse macros
\usepackage[top=1in,bottom=1in,left=1in,right=1in]{geometry} % 8.5" x 11" pages with 1 inch margins
\usepackage[pdftex, bookmarks, colorlinks, breaklinks]{hyperref} % prettier hyperlinks
\usepackage[usenames,dvipsnames,svgnames,table]{xcolor} % defines colors for text and tikz graphics
\definecolor{darkred}{rgb}{0.8,0.1,0.2} % for hyperlinks
\definecolor{darkblue}{rgb}{0.2,0.1,0.7} % for hyperlinks
\hypersetup{linkcolor=darkred,citecolor=blue,filecolor=dullmagenta,urlcolor=darkblue} % colors for links
\usepackage[none]{hyphenat} % prettier hyphenating
\raggedright \parskip4pt  \parindent0pt 
\usepackage{array} % tables with paragraphs of set widths
\renewcommand{\arraystretch}{1.3} % makes tables more legible
\newcolumntype{L}[1]{>{\raggedright\let\newline\\\arraybackslash\hspace{0pt}}p{#1}}
\newcolumntype{C}[1]{>{\centering\let\newline\\\arraybackslash\hspace{0pt}}p{#1}}
\newcolumntype{R}[1]{>{\raggedleft\let\newline\\\arraybackslash\hspace{0pt}}p{#1}}
\usepackage{rotating} % rotating figures and tables, provides sidewaystable and sidewaysfigure
\usepackage{lipsum} % text testing


%%%%%%% DOCUMENT MANAGEMENT %%%%%%%%%%%%
% To do notes and commenting 
\usepackage{comment}

% View instructor notes 
\newif\ifinstructor 
 \instructortrue  % view as instructor 
% \instructorfalse  % view as student
  
%%%%%%% SECTION FORMATTING %%%%%%%%%%%%
% sections
\usepackage{titlesec}
\titleformat{\subsection}[block]{\Large \bfseries \filcenter}{}{0em}{}
\titleformat{\subsubsection}[block]{\large \scshape\filcenter}{}{0em}{}
\newcommand{\handout}{\subsubsection}
\newcommand\header[1]{\vspace*{4pt}\par {\large {\bf #1}}\par}
\newcommand\about{\textasciitilde}

% itemize - second layer is an open circle instead of dash
\def\labelitemii{$\circ$}

% table colors - mostly for fun
\definecolor{yellow}{RGB}{255, 255, 0}
\definecolor{red}{RGB}{226, 30, 60}
\definecolor{orange}{RGB}{255, 159, 12}
\definecolor{green}{RGB}{16, 168, 112}
\definecolor{blue}{RGB}{1,200,255}
\definecolor{periwinkle}{RGB}{200,200,255}
\definecolor{lightteal}{RGB}{200,250,250}
\definecolor{purple}{RGB}{113, 1, 232}
%\definecolor{pink}{RGB}{255,70,192}
\definecolor{pink}{RGB}{232,1,193}
\definecolor{gray}{RGB}{100, 100, 100}

% instructor notes
\ifinstructor 
\newenvironment{bignote}[1][Instructor note]% default note is an "Instructor Note"
	{\begin{mdframed}\raggedright{\bf #1.~}}
	{\end{mdframed}}  
\else \excludecomment{bignote}
\fi

\ifinstructor
\newcommand\smallnote[1]
	{\begin{mdframed}\raggedright  {\bf Instructor note.} {#1} \end{mdframed}}
\else \newcommand\smallnote[1]{}
\fi

\ifinstructor  \usepackage{todonotes} 
\else \usepackage[disable]{todonotes}
\fi

% in-class task
\newenvironment{task}
	{\begin{mdframed}[linecolor=lightgray, linewidth=3pt]\raggedright}
	{\end{mdframed}}

%%% GRAPHICS / TIKZ %%%%%%%%%%%
\graphicspath{{Images/}}

\usepackage{tikz}
\usepackage{tkz-euclide} % tikz package for Euclidean geometry
\usepackage{siunitx} % typesetting quantities
\usepackage{pgfplots} %\pgfplotsset{compat=1.13} % plotting graphs
\usetikzlibrary{calc} % calculations within tikz
\usetkzobj{all} % needed for tkz-euclide package
 
%%%%% MATH NOTATION %%%%%%%%

\newcommand\tn{\textnormal}

% systems
\newcommand{\R}{\mathbb{R}}
\newcommand{\C}{\mathbb{C}}
\newcommand{\Q}{\mathbb{Q}}
\newcommand{\N}{\mathbb{N}}
\newcommand{\Z}{\mathbb{Z}}

% notation tweaking
\renewcommand\phi\varphi  % normal \phi looks too much like the empty set.
\renewcommand\subset\subseteq 
\renewcommand\supset\supseteq  % to be careful about strict subsets and nonstrict subsets
\newcommand\st{:}

% divisibility
\newcommand\divides{\;|\;}
\newcommand\notdivides{\hspace*{-2pt}\not |\;}

% trig and geometry
\newcommand\degrees{^\circ}

%%%%%%%%%% THEOREMS AND RELATED STRUCTURES %%%%%
\renewcommand\emph[1]{\underline{\bf{#1}}} % terminology

\newtheorem{theorem}{Theorem}[section]
\newtheorem{proposition}[theorem]{Proposition}
\newtheorem{lemma}[theorem]{Lemma}
\newtheorem{corollary}[theorem]{Corollary}
\newtheorem{claim}{Claim}

\theoremstyle{definition}
\newtheorem{definition}[theorem]{Definition}
\newtheorem{example}[theorem]{Example}
\newtheorem{problem}[theorem]{Problem}
\newtheorem{conjecture}[theorem]{Conjecture}
\newtheorem{question}[theorem]{Question}
\newtheorem{remark}[theorem]{Remark}
\newtheorem{case}{Case}

\newtheorem*{theorem*}{Theorem}
\newtheorem*{example*}{Example}
\newtheorem*{question*}{Question}
\newtheorem*{claim*}{Claim}
\newtheorem*{definition*}{Definition}

\newenvironment{solution}{{\it Solution.} }{\hfill {\color{lightgray}$\blacksquare$}}

\newcommand\qedpart[1]{ \hfill \framebox(6,6){\tiny #1}}
\renewcommand\qed{\hfill \framebox(6,6){}}
%%%%%%%%%%%%%%%%%%%%%%%%%%%%%%%% 	
%%%%%%%% DOCUMENT BEGINS %%%%%%%%%%%%
%%%%%%%%%%%%%%%%%%%%%%%%%%%%%%%% 

\begin{document}

%%%%%% COVER PAGE %%%%%%%%%%%%%%% 
\pagenumbering{gobble} % no page number
\maketitle
\ifthenelse{\equal{\coursegraphic}{}} % insert course graphic if one exists
	{}
	{\begin{center}\includegraphics[width=3in]{\coursegraphic}\end{center}}
	
\vfill 
% copyleft
\begin{center} \includegraphics[width=1in]{by-nc-sa.png} \end{center}
\footnotesize{ This work is licensed under a Creative Commons Attribution-ShareAlike 3.0 Unported License. }

 % acknowledgments 
\footnotesize{
The Mathematics Of Doing, Understand, Learning, and Educating Secondary Schools (\MODULES) project is partially supported by funding from a collaborative grant of the National Science Foundation under Grant Nos.~DUE-1726707,1726804, 1726252, 1726723, 1726744, and 1726098.  Any opinions, findings, and conclusions or recommendations expressed in this material are those of the authors and do not necessarily reflect the views of the National Science Foundation.}
\newpage
%%%%%% TABLE OF CONTENTS %%%%%%%%%%%%%%% 	
\thispagestyle{plain} \pagenumbering{roman}  
\listoftodos
\tableofcontents
\newpage \pagenumbering{arabic}
%%%%%%%%%%%%%%%%%%%%%%%%%%%%%%%% 
%%%%%%%%%%%%%%%%%%%%%%%%%%%%%%%% 	
%%%%%% PART 1 %%%%%%%%%%%%%%%%%%%%%	
%%%%%%%%%%%%%%%%%%%%%%%%%%%%%%%% 
%%%%%%%%%%%%%%%%%%%%%%%%%%%%%%%% 	
\newpage 

\part{Introduction to Complex Numbers}

\section{Basics of Complex Numbers}

\subsection{Defining Complex Numbers}

First, a bit of review. The set of natural numbers, $\mathbb{N}$, consists of all of the counting numbers:
\[ \mathbb{N} = \{ 0, 1, 2, 3, \dots \}.\]
The set of integers, $\mathbb{Z}$, consists of the natural numbers and their additive inverses:
\[ \mathbb{Z} = \{ \dots,-3,-2,-1,0, 1, 2, 3, \dots \}.\]
The rational numbers are those numbers that can be written as a fraction:
\[ \mathbb{Q} = \{ \frac{a}{b} \mid a, b\in \mathbb{Z},\, b \neq 0 \}.\]
Some numbers cannot be written as a ratio of two integers, and we call those numbers irrational. For example, $\sqrt{2}$ is not a rational number; it is
irrational. But, finally, we have the set of real numbers, $\mathbb{R}$, which consists of all the rational numbers together with all of the irrational numbers. 
So, we have the following containments:
\[ \mathbb{N} \subseteq \mathbb{Z} \subseteq \mathbb{Q} \subseteq \mathbb{R}.\]
We often model the real numbers on a line, called the real number line. 
\begin{center}
  INSERT IMAGE
\end{center}
Real numbers can added, subtracted, multiplied and divided. As suggested by the number line model, there is also an ordering of the real 
numbers with the following properties:
\begin{itemize}
  \item For any real numbers $a$ and $b$, if $a\leq b$ and $b\leq a$, then $a=b$. (Antisymmetry)
  \item For any real numbers $a$, $b$, and $c$, if $a \leq b$ and $b\leq c$, then $a\leq c$. (Transitivity)
  \item For any real numbers $a$ and $b$, $a\leq b$ or $b\leq a$. (Totality)
\end{itemize}

These properties fit our intuition for how objects on a number line should behave, and for that reason we call an ordering with these properties a 
\emph{linear order}. So, we say the real numbers are ``linearly orderd.'' The real numbers also have a very important property called ``completeness,'' 
which allows us to conclude that there are no ``gaps'' in the real numbers. Although the real numbers are complete in this sense, it turns out that there 
are some problems which can't be solved with real numbers.

\begin{task}
  Find two numbers whose sum is 10 and whose product is 16.

  Find two numbers whose sum is 10 and whose product is 25.

  Find two numbers whose sum is 10 and whose product is 40.
\end{task}

In the last task in the inquiry above, the equation
\[ x^2 - 10x + 40 = 0\]
is found to have no real solutions. So are there two numbers which satisfy the requirements?  Well, there are no \emph{real} numbers that work, but
we can find two \emph{complex numbers}: $5 + i\sqrt{15}$ and $5-i\sqrt{15}$. In the sections that follow we will introduce complex numbers and study their 
properties.

\subsubsection{Ordered Pairs}

You may know that complex numbers have the form $a + bi$ where $a$ and $b$ are real numbers and $i = \sqrt{-1}$. And, you may also know that we can model 
complex numbers in the plane. We will begin our study of complex numbers by viewing them as points or vectors in the plane. Thus, we can say that a 
\emph{complex} number $z$ is an orderd pair
\[ z = (a,b)\]
of real numbers $a$ and $b$. This representation allows us to very naturally graph complex numbers in the Cartesian plane:

\begin{center}
  INSERT IMAGE
\end{center}

Also, in this representation, it is natural to identify a complex number of the form $(a,0)$ with the real number $a$. Thus we identify the complex number
$(3,0)$ with the real number 3 and we identify the complex number $\left( \frac{1}{3},0 \right)$ with the real number $\frac{1}{3}$. In this way, 
the complex numbers contain the real numbers as a subset. 

Thus, if we denote the set of complex numbers by the symbol $\mathbb{C}$, then we have,
\[ \mathbb{N} \subseteq \mathbb{Z} \subseteq \mathbb{Q} \subseteq \mathbb{R} \subseteq \mathbb{C}.\]

By the way, complex numbers of the form $(0,b)$ are called \emph{pure imaginary numbers}, or just \emph{imaginary numbers}.

Consistent with this, if we are given a complex number $z=(a,b)$ we call $a$ the \emph{real part} of $z$ and we call $b$ the \emph{imaginary part} of $z$. The 
notation for this is 
\[ \text{Re }z = a \text{ and } \text{Im }=b.\]
Two complex numbers are said to be equal when they have the same real parts and the same imaginary parts. That is (just like ordered pairs), we say
\[ (a,b) = (c,d) \text{ if and only if } a=c \text{ and } b=d. \]

Now, notice that
\[ z = (a,b) = a \cdot (1,0) + b \cdot (0,1). \]
We already know that we can identify the complex number $(1,0)$ with the real number 1. We will also \emph{let $i$ denote the pure imaginary number $(0,1)$.}
Thus, we can write
\begin{align*}
 z &= (a,b)\\ 
   &= a \cdot (1,0) + b \cdot (0,1)\\
   &= a\cdot 1 + b\cdot i\\
   &= a + bi
\end{align*}

Many students are more familiar with the form $z = a + bi$ for the complex number $(a,b)$. Both representations are useful. We will usually
use the form $z = a+bi$, but we will also often use the form $z=(a,b)$. The notation $a+bi$ is very flexible, but there are some conventions. For example,
$5+3i$, $5+i3$, $3i+5$ and $i3+5$ are all equal, but we usually write it as $5+3i$. Instead of writing something like $8+1i$, we would usually 
just write $8+i$. And, rather than $3+0i$ or $0+0i$ we typically just write $3$ or $0$, respectively.

Before we move on to discuss arithemtic with complex numbers, let us briefly discuss the issue of the ``size'' of a complex number. We are very
familiar with this concept when it comes to real numbers. For example, we know that $\pi$ is larger than 3. We know that $-4$ is less than $-1$. As we will
see, it turns out that we cannot order

TODO: ordering, magnitude, have students figure out formula for magnitude of a complex number based on formula for magnitude of a vector.

\subsection{Addition and Subtraction of Complex Numbers}

Complex numbers are identified with points in the plane for some very good reasons. One of those reasons is that addition works as expected: we
add them \emph{component wise}. That is, given complex numbers $z_1=(a,b)$ and $z_2 = (c,d)$ we add them as follows:
\begin{align*}
  z_1 + z_2 &= (a,b)+(c,d)\\
            &= (a+c, b+d)
\end{align*}
If we had represented these complex numbers as $z_1=a+bi$ and $z_2=c+di$ then addition would work as follows
\begin{align*}
  z_1 + z_2 &= (a+bi)+(c+di)\\
            &= (a+c) + (b+d)i
\end{align*}

\begin{task}
  In any number system, we say a number $\mathbf{n}$ is an \emph{additive identity} if $x + \mathbf{n} = \mathbf{n} + x = x$ for any number $x$ in that 
  number system.
  \begin{itemize}
    \item What is the additive identity of the natural numbers? What about the integers, rationals, and reals?
    \item What is the additive identity in the set of complex numbers $\mathbb{C}$? Write it in two ways.
  \end{itemize}
  Suppose $x$ is a number of any type. It can be an integer, rational, real or complex number. We define the additive inverse of $x$ to be the number $y$ so
  that $x+y = \textbf{0}$. 
\begin{itemize}
  \item Given a complex number $z = a+bi$, what is its additive inverse? 
  \item How would you write $z$ and its additive inverse using ordered pairs?
\end{itemize}
  How would you define subtraction of complex numbers? Write the definition using both the $a+bi$ representation and the orderd pair representation.
\end{task}

\subsection{Multiplication of Complex Numbers}

\subsubsection{i}

\subsubsection{Complex Conjugate}

\subsubsection{Modulus}



\part{Representations of Complex Numbers}

\part{Roots and polynomials}



\end{document}



