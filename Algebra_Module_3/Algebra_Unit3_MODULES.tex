%%%%% META DATA %%%%%%%%%%%%
\newcommand\Course{Algebra for Secondary Mathematics Teaching} % e.g., Algebra, Geometry, Modeling, Statistics
\newcommand\Location{MODULE($\textnormal{S}^2$)} % affiliation and course
\newcommand\Term{Spring 2018} % term taught

 \newcommand\MODULES{$\textnormal{MODULE}(\textnormal{S}^2)$}
 \title{{\normalsize{Mathematics Of Doing, Understand, Learning, and Educating Secondary Schools} }\\  $\;$ \\ \MODULES: \\  \Course}
 \author{Adapted for \Location} 
 \date{Version \Term} 
 
 % Input a course graphic or leave the { } empty:
\newcommand\coursegraphic{DoubleRectangles.png} 
 
 %%%%%%%DOCUMENT FORMATTING %%%%%%%%%%
\documentclass[11pt]{article}
\linespread{1.03}% 6 lpi http://tex.stackexchange.com/questions/23824/6-lines-in-one-inch
\parskip6pt
\usepackage{amsmath, amsthm, amsfonts, amssymb, mathpazo, url, graphicx, stackrel, mdwlist, enumitem, mdframed, ifthen}
	% usual suspects, palatino, hyperlink capability, PDF graphics, symbol stacking, list customizations, boxes, ifthenelse macros
\usepackage[top=1in,bottom=1in,left=1in,right=1in]{geometry} % 8.5" x 11" pages with 1 inch margins
\usepackage[pdftex, bookmarks, colorlinks, breaklinks]{hyperref} % prettier hyperlinks
\usepackage[usenames,dvipsnames,svgnames,table]{xcolor} % defines colors for text and tikz graphics
\definecolor{darkred}{rgb}{0.8,0.1,0.2} % for hyperlinks
\definecolor{darkblue}{rgb}{0.2,0.1,0.7} % for hyperlinks
\hypersetup{linkcolor=darkred,citecolor=blue,filecolor=dullmagenta,urlcolor=darkblue} % colors for links
\usepackage[none]{hyphenat} % prettier hyphenating
\raggedright \parskip4pt  \parindent0pt 
\usepackage{array} % tables with paragraphs of set widths
\renewcommand{\arraystretch}{1.3} % makes tables more legible
\newcolumntype{L}[1]{>{\raggedright\let\newline\\\arraybackslash\hspace{0pt}}p{#1}}
\newcolumntype{C}[1]{>{\centering\let\newline\\\arraybackslash\hspace{0pt}}p{#1}}
\newcolumntype{R}[1]{>{\raggedleft\let\newline\\\arraybackslash\hspace{0pt}}p{#1}}
\usepackage{rotating} % rotating figures and tables, provides sidewaystable and sidewaysfigure
\usepackage{lipsum} % text testing


%%%%%%% DOCUMENT MANAGEMENT %%%%%%%%%%%%
% To do notes and commenting 
\usepackage{comment}

% View instructor notes 
\newif\ifinstructor 
 \instructortrue  % view as instructor 
% \instructorfalse  % view as student
  
%%%%%%% SECTION FORMATTING %%%%%%%%%%%%
% sections
\usepackage{titlesec}
\titleformat{\subsection}[block]{\Large \bfseries \filcenter}{}{0em}{}
\titleformat{\subsubsection}[block]{\large \scshape\filcenter}{}{0em}{}
\newcommand{\handout}{\subsubsection}
\newcommand\header[1]{\vspace*{4pt}\par {\large {\bf #1}}\par}
\newcommand\about{\textasciitilde}

% itemize - second layer is an open circle instead of dash
\def\labelitemii{$\circ$}

% table colors - mostly for fun
\definecolor{yellow}{RGB}{255, 255, 0}
\definecolor{red}{RGB}{226, 30, 60}
\definecolor{orange}{RGB}{255, 159, 12}
\definecolor{green}{RGB}{16, 168, 112}
\definecolor{blue}{RGB}{1,200,255}
\definecolor{periwinkle}{RGB}{200,200,255}
\definecolor{lightteal}{RGB}{200,250,250}
\definecolor{purple}{RGB}{113, 1, 232}
%\definecolor{pink}{RGB}{255,70,192}
\definecolor{pink}{RGB}{232,1,193}
\definecolor{gray}{RGB}{100, 100, 100}

% instructor notes
\ifinstructor 
\newenvironment{bignote}[1][Instructor note]% default note is an "Instructor Note"
	{\begin{mdframed}\raggedright{\bf #1.~}}
	{\end{mdframed}}  
\else \excludecomment{bignote}
\fi

\ifinstructor
\newcommand\smallnote[1]
	{\begin{mdframed}\raggedright  {\bf Instructor note.} {#1} \end{mdframed}}
\else \newcommand\smallnote[1]{}
\fi

\ifinstructor  \usepackage{todonotes} 
\else \usepackage[disable]{todonotes}
\fi

% in-class task
\newenvironment{task}
	{\begin{mdframed}[linecolor=lightgray, linewidth=3pt]\raggedright}
	{\end{mdframed}}

%%% GRAPHICS / TIKZ %%%%%%%%%%%
\graphicspath{{Images/}}

\usepackage{tikz}
\usepackage{tkz-euclide} % tikz package for Euclidean geometry
\usepackage{siunitx} % typesetting quantities
\usepackage{pgfplots} %\pgfplotsset{compat=1.13} % plotting graphs
\usetikzlibrary{calc} % calculations within tikz
\usetkzobj{all} % needed for tkz-euclide package
 
%%%%% MATH NOTATION %%%%%%%%

\newcommand\tn{\textnormal}

% systems
\newcommand{\R}{\mathbb{R}}
\newcommand{\C}{\mathbb{C}}
\newcommand{\Q}{\mathbb{Q}}
\newcommand{\N}{\mathbb{N}}
\newcommand{\Z}{\mathbb{Z}}

% notation tweaking
\renewcommand\phi\varphi  % normal \phi looks too much like the empty set.
\renewcommand\subset\subseteq 
\renewcommand\supset\supseteq  % to be careful about strict subsets and nonstrict subsets
\newcommand\st{:}

% divisibility
\newcommand\divides{\;|\;}
\newcommand\notdivides{\hspace*{-2pt}\not |\;}

% trig and geometry
\newcommand\degrees{^\circ}

%%%%%%%%%% THEOREMS AND RELATED STRUCTURES %%%%%
\renewcommand\emph[1]{\underline{\bf{#1}}} % terminology

\newtheorem{theorem}{Theorem}[section]
\newtheorem{proposition}[theorem]{Proposition}
\newtheorem{lemma}[theorem]{Lemma}
\newtheorem{corollary}[theorem]{Corollary}
\newtheorem{claim}{Claim}

\theoremstyle{definition}
\newtheorem{definition}[theorem]{Definition}
\newtheorem{example}[theorem]{Example}
\newtheorem{problem}[theorem]{Problem}
\newtheorem{conjecture}[theorem]{Conjecture}
\newtheorem{question}[theorem]{Question}
\newtheorem{remark}[theorem]{Remark}
\newtheorem{case}{Case}

\newtheorem*{theorem*}{Theorem}
\newtheorem*{example*}{Example}
\newtheorem*{question*}{Question}
\newtheorem*{claim*}{Claim}
\newtheorem*{definition*}{Definition}

\newenvironment{solution}{{\it Solution.} }{\hfill {\color{lightgray}$\blacksquare$}}

\newcommand\qedpart[1]{ \hfill \framebox(6,6){\tiny #1}}
\renewcommand\qed{\hfill \framebox(6,6){}}
%%%%%%%%%%%%%%%%%%%%%%%%%%%%%%%% 	
%%%%%%%% DOCUMENT BEGINS %%%%%%%%%%%%
%%%%%%%%%%%%%%%%%%%%%%%%%%%%%%%% 

\begin{document}

%%%%%% COVER PAGE %%%%%%%%%%%%%%% 
\pagenumbering{gobble} % no page number
\maketitle
\ifthenelse{\equal{\coursegraphic}{}} % insert course graphic if one exists
	{}
	{\begin{center}\includegraphics[width=3in]{\coursegraphic}\end{center}}
	
\vfill 
% copyleft
\begin{center} \includegraphics[width=1in]{by-nc-sa.png} \end{center}
\footnotesize{ This work is licensed under a Creative Commons Attribution-ShareAlike 3.0 Unported License. }

 % acknowledgments 
\footnotesize{
The Mathematics Of Doing, Understand, Learning, and Educating Secondary Schools (\MODULES) project is partially supported by funding from a collaborative grant of the National Science Foundation under Grant Nos.~DUE-1726707,1726804, 1726252, 1726723, 1726744, and 1726098.  Any opinions, findings, and conclusions or recommendations expressed in this material are those of the authors and do not necessarily reflect the views of the National Science Foundation.}
\newpage
%%%%%% TABLE OF CONTENTS %%%%%%%%%%%%%%% 	
\thispagestyle{plain} \pagenumbering{roman}  
\listoftodos
\tableofcontents
\newpage \pagenumbering{arabic}
%%%%%%%%%%%%%%%%%%%%%%%%%%%%%%%% 
%%%%%%%%%%%%%%%%%%%%%%%%%%%%%%%% 	
%%%%%% PART 1 %%%%%%%%%%%%%%%%%%%%%	
%%%%%%%%%%%%%%%%%%%%%%%%%%%%%%%% 
%%%%%%%%%%%%%%%%%%%%%%%%%%%%%%%% 	
\newpage 

\part{Introduction to Fields}

\section{Fields and Other Algebraic Structures}

%%Day 1

In this section we will begin our study of \emph{fields}. You've already encountered fields in your mathematical studies: the set of rational numbers $\mathbb{Q}$ and the set of real numbers $\mathbb{R}$ are fields, 
as is the set of complex numbers $\mathbb{C}$. The sets $\mathbb{Q}$, $\mathbb{R}$ and $\mathbb{C}$ are different in many ways, but here we will focus on the ways in which they are similar. We will also see that there
are fields that are different from these three in some very important ways.

Consider the equation
      \[ 3x + 8 = 14. \]
It's not hard to see that the solution to this equation is $x=2$: $3(2)+8 = 14$. Let's us solve this equation step-by-step, justifying each step along the way. First we will subtract 8 from both sides:
\[ (3x+8)-8 = 14 -8.\]
(Note that we could also view this as adding -8 to both sides. The number $-8$ is known as the \emph{additive inverse} of 8.) Applying the associative law on the left-hand side gives
\[ 3x + (8-8) = 6.\]
We know that $8-8=0$ so we have
\[ 3x + 0 = 6.\]
The number 0 is an \emph{additive identity}. That means adding 0 returns the value we added it to. So we have
\[ 3x = 6.\]
We now multiply each side by $1/3$ to obtain
\[ \frac{1}{3}(3x) = \frac{1}{3}\cdot 6. \]
Multiplication is associative, so we can write this as
\[ \left( \frac{1}{3}\cdot 3 \right)x = 2. \]
The number $1/3$ is the \emph{multiplicative inverse} of $3$, meaning that $\frac{1}{3}\cdot 3$ is equal to the \emph{multiplicative identity}; that is, $\frac{1}{3}\cdot 3 = 1$. Thus we have
\[ 1x = 2.\]
The number 1 is the \emph{multiplicative identity} meaning that $1x = x$. So we conclude that
\[ x = 2.\]

Let us analyze this situation more carefully. First note that the equation $3x+8 = 14$ uses two operations, called addition and multiplication. 
(Subtraction can always be defined in terms of addition, and division can be defined in terms of multiplication.)
We used some familiar properites of addition and multiplication such as associativity of addition and multiplication.

Above we multplied by $1/3$ at point in the solution. Since $1/3$ is a rational number, we say that we solved this equation ``over the
rationals.'' But, notice that in this example we didn't really need to do this. Next we give a solution to the equation $3x+8=14$ ``over the
integers.'' We begin the same way:
\begin{align*}
  3x + 8 &= 14\\
 (3x+8)-8 &= 14 -8\\
 3x + (8-8) &= 6\\
 3x + 0 &= 6\\
 3x &= 6.
\end{align*}
Next we observe that $6 = 3(2)$ so we have
\[ 3x = 3(2).\]
One can prove that in the integers that if $a$, $b$, and $c$ are integers and $ab=ac$ then $b=c$. Using just that fact, we can conclude that
\[ x = 2.\]

\begin{task}
  \begin{itemize}
    \item Prove that if $a$, $b$, and $c$ are integers and $ab = ac$ then $b=c$. Remember - division is not allowed, we want to do this
      proof entirely in the integers.
    \item Can you solve the equation $3x+8=14$ over the natural numbers? (Here you're not allowed to use additive inverses!)
  \end{itemize}
\end{task}

We call $0$ an additive identity because for any number $n$, $n+0 = 0 + n = n$.

\begin{task}
  \begin{itemize}
    \item Consider the collection of all $2\times 2$ matricies whose entries are real numbers. Write down the additive identity of this set.
    \item How would you define the general notion of a \emph{multiplicative identity}? What is a multiplicative identity in $\mathbb{Q}$?
    \item Is there a multiplicative identity for the set of all $2\times 2$ matricies with real entries?
  \end{itemize}
\end{task}

Once we have a notion of an additive identity, we can define the notion of an additive inverse. We say that 
a number $b$ is an additive inverse of a number $a$ if and only if $a+b=b+a = 0$. If $b$ is an additive
inverse of $a$ we write $b = -a$. Note that $-a$ may be positive or negative. For example,
the additive inverse of $4$ is $-4$, but the additve inverse of $-5$ is $5$.

\begin{task}
  How would you define the notion of a \emph{multiplicative inverse}? Give an example of a number $a$ and its 
  multiplicative inverse $b$.
\end{task}

A \emph{field} $\mathbb{F}$ is a collection of mathematical objects (possibly numbers, matrices, functions, etc.) with two operations, called
addition ($+$) and multiplication ($\cdot$), in which we can always solve an equation of the form
\[ ax + b = c\]
where $a,b,c\mathbb{F}$ and $a \neq 0$. The properties we need to make this happen are given in the following definition.

\begin{definition} A field $\mathbb{F}$ is a nonempty set together with two operations addition $+$ and multiplication $\cdot$ which satisfy the following
  properties, called the field axioms:
  \begin{enumerate}
    \item If $a,b\in \mathbb{F}$, there is a unique $a+b \in \mathbb{F}$.
    \item Addition is associative. That is, if $a,b,c\in\mathbb{F}$, then
      \[ (a+b)+c = a + (b+c).\]
    \item Addition is commutative. That is, if $a,b\in\mathbb{F}$, then
      \[ a+b = b+a.\]
    \item There is an additive identity in $\mathbb{F}$.
    \item If $a\in \mathbb{F}$, then $a$ has an additive inverse in $\mathbb{F}$.
    \item If $a,b\in\mathbb{F}$, then there is a unique $a\cdot b \in \mathbb{F}$.
    \item Multiplication is associative.
    \item Multiplication is commutatitive.
    \item There is a multiplicative identity in $\mathbb{F}$.
    \item If $a\in \mathbb{F}$ and $a\neq 0$, then there is a multiplicative inverse of $a$ in $\mathbb{F}$.
    \item Multiplication distributes over addition. That is, if $a,b,c\in\mathbb{F}$, then 
      \[ a\cdot (b+c) = a\cdot b + a\cdot c.\]
    \item The additive identity does not equal the multiplicative identity.
  \end{enumerate}
\end{definition}

Of course, you have seen fields before: the rational numbers $\mathbb{Q}$ and the real numbers $\mathbb{R}$ are both fields under their usual operations of
addition and multplication. In fact, $\mathbb{Q}$ is a \emph{subfield} of $\mathbb{R}$.

\subsection{More on Identities and Inverses}

We all know that in the rational numbers there is only one additive identity: the number 0. But could it be that there is a field with more than one
additive identity? We have the following proposition:

\begin{proposition}
  In any field $\mathbb{F}$, then additive identity is unique.
\end{proposition}
\begin{proof}
  Suppose that we have additive identities $0$ and $z$ in $\mathbb{F}$. Since $0$ is an additive identity, we know that
  \[ 0 + z = z.\]
  But since $z$ is also an additive identity, we also know that
  \[ 0 + z = 0.\]
  So, we have that
  \[ z = 0 + z = 0.\]
  This proves that the additive identity in any field is unique.
\end{proof}

There are a couple of observations to make about this prove. First, a good general strategy for proving that something is unique is to assume that there are
two of them and then prove that they are equal.  If needed, you can also assume that your two proposed objects are not equal and derive a contratiction, but
notice that we did not need to do that in the proof above. Second, observe that besides using the definition of an additive identity, the only other property we
used to prove the proposition above is that addition is commutatitive.

Since the additive identity in any field is unique, we will almost always use the usual symbol $0$ to represent it, unless we have reason not to. 

\begin{task}
  Use the proof above as a model to show that in any field the multiplicative identity is unique.
\end{task}

Similarly, since the multplicative identity in any field is unique, we will almost always use the usual symbol $1$ to represent it. 

There is a similar fact to observe with respect to additive and multiplicative inverses.  For example, there is only one rational number whose sum with
$-\frac{1}{2}$ is 0, namely $\frac{1}{2}$. Similarly, there is only one rational number whose product with $-\frac{1}{2}$ is $1$, namely $-2$. We have:

\begin{proposition}
  If $\mathbb{F}$ is a field and $a\in \mathbb{F}$, then its additive inverse is unique to it.
\end{proposition}
\begin{proof}
  Suppose that $\mathbb{F}$ is a field and that $a\in\mathbb{F}$. We want to prove that there is only one element $b\in\mathbb{F}$ so that
  \[ a+b=b+a=0.\]
  To this end, suppose that there are two such elements $b,c\in\mathbb{F}$. Then we have both:
  \begin{align*}
    a + b = b + a &= 0\\
    a + c = c + a &= 0
  \end{align*}
  Consider the sum $b + a + c$. On one hand we have
  \[ b + a + c = (b+a) + c = 0 + c = c.\]
  On the other hand we have
  \[ b + a + c = b+ (a + c) = b + 0 = b.\]
  Thus we conclude that $c=b$ and that every element in a field has a unique additive inverse.
\end{proof}

\begin{proposition}
  If $a\in\mathbb{F}$, then $a\cdot 0 = 0 \cdot a = 0$.
\end{proposition}

\begin{proposition}
  Suppose $a,b\in\mathbb{F}$. Then
  \begin{enumerate}
    \item $-(-a) = a$
    \item $-a = (-1)a$
    \item $-(a+b)=(-a) + (-b)$
    \item $-(a\cdot b) = (-a)\cdot b = a\cdot (-b)$.
  \end{enumerate}
\end{proposition}

\subsection{Ordered Fields}

\part{Constructible Numbers}

\part{Three Famous Problems}



\end{document}



