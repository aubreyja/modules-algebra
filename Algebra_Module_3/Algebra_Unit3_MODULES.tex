%%%%% META DATA %%%%%%%%%%%%
\newcommand\Course{Algebra for Secondary Mathematics Teaching} % e.g., Algebra, Geometry, Modeling, Statistics
\newcommand\Location{MODULE($\textnormal{S}^2$)} % affiliation and course
\newcommand\Term{Spring 2018} % term taught

 \newcommand\MODULES{$\textnormal{MODULE}(\textnormal{S}^2)$}
 \title{{\normalsize{Mathematics Of Doing, Understand, Learning, and Educating Secondary Schools} }\\  $\;$ \\ \MODULES: \\  \Course}
 \author{Adapted for \Location} 
 \date{Version \Term} 
 
 % Input a course graphic or leave the { } empty:
\newcommand\coursegraphic{DoubleRectangles.png} 
 
 %%%%%%%DOCUMENT FORMATTING %%%%%%%%%%
\documentclass[11pt]{article}
\linespread{1.03}% 6 lpi http://tex.stackexchange.com/questions/23824/6-lines-in-one-inch
\parskip6pt
\usepackage{amsmath, amsthm, amsfonts, amssymb, mathpazo, url, graphicx, stackrel, mdwlist, enumitem, mdframed, ifthen}
	% usual suspects, palatino, hyperlink capability, PDF graphics, symbol stacking, list customizations, boxes, ifthenelse macros
\usepackage[top=1in,bottom=1in,left=1in,right=1in]{geometry} % 8.5" x 11" pages with 1 inch margins
\usepackage[pdftex, bookmarks, colorlinks, breaklinks]{hyperref} % prettier hyperlinks
\usepackage[usenames,dvipsnames,svgnames,table]{xcolor} % defines colors for text and tikz graphics
\definecolor{darkred}{rgb}{0.8,0.1,0.2} % for hyperlinks
\definecolor{darkblue}{rgb}{0.2,0.1,0.7} % for hyperlinks
\hypersetup{linkcolor=darkred,citecolor=blue,filecolor=dullmagenta,urlcolor=darkblue} % colors for links
\usepackage[none]{hyphenat} % prettier hyphenating
\raggedright \parskip4pt  \parindent0pt 
\usepackage{array} % tables with paragraphs of set widths
\renewcommand{\arraystretch}{1.3} % makes tables more legible
\newcolumntype{L}[1]{>{\raggedright\let\newline\\\arraybackslash\hspace{0pt}}p{#1}}
\newcolumntype{C}[1]{>{\centering\let\newline\\\arraybackslash\hspace{0pt}}p{#1}}
\newcolumntype{R}[1]{>{\raggedleft\let\newline\\\arraybackslash\hspace{0pt}}p{#1}}
\usepackage{rotating} % rotating figures and tables, provides sidewaystable and sidewaysfigure
\usepackage{lipsum} % text testing


%%%%%%% DOCUMENT MANAGEMENT %%%%%%%%%%%%
% To do notes and commenting 
\usepackage{comment}

% View instructor notes 
\newif\ifinstructor 
 \instructortrue  % view as instructor 
% \instructorfalse  % view as student
  
%%%%%%% SECTION FORMATTING %%%%%%%%%%%%
% sections
\usepackage{titlesec}
\titleformat{\subsection}[block]{\Large \bfseries \filcenter}{}{0em}{}
\titleformat{\subsubsection}[block]{\large \scshape\filcenter}{}{0em}{}
\newcommand{\handout}{\subsubsection}
\newcommand\header[1]{\vspace*{4pt}\par {\large {\bf #1}}\par}
\newcommand\about{\textasciitilde}

% itemize - second layer is an open circle instead of dash
\def\labelitemii{$\circ$}

% table colors - mostly for fun
\definecolor{yellow}{RGB}{255, 255, 0}
\definecolor{red}{RGB}{226, 30, 60}
\definecolor{orange}{RGB}{255, 159, 12}
\definecolor{green}{RGB}{16, 168, 112}
\definecolor{blue}{RGB}{1,200,255}
\definecolor{periwinkle}{RGB}{200,200,255}
\definecolor{lightteal}{RGB}{200,250,250}
\definecolor{purple}{RGB}{113, 1, 232}
%\definecolor{pink}{RGB}{255,70,192}
\definecolor{pink}{RGB}{232,1,193}
\definecolor{gray}{RGB}{100, 100, 100}

% instructor notes
\ifinstructor 
\newenvironment{bignote}[1][Instructor note]% default note is an "Instructor Note"
	{\begin{mdframed}\raggedright{\bf #1.~}}
	{\end{mdframed}}  
\else \excludecomment{bignote}
\fi

\ifinstructor
\newcommand\smallnote[1]
	{\begin{mdframed}\raggedright  {\bf Instructor note.} {#1} \end{mdframed}}
\else \newcommand\smallnote[1]{}
\fi

\ifinstructor  \usepackage{todonotes} 
\else \usepackage[disable]{todonotes}
\fi

% in-class task
\newenvironment{task}
	{\begin{mdframed}[linecolor=lightgray, linewidth=3pt]\raggedright}
	{\end{mdframed}}

%%% GRAPHICS / TIKZ %%%%%%%%%%%
\graphicspath{{Images/}}

\usepackage{tikz}
\usepackage{tkz-euclide} % tikz package for Euclidean geometry
\usepackage{siunitx} % typesetting quantities
\usepackage{pgfplots} %\pgfplotsset{compat=1.13} % plotting graphs
\usetikzlibrary{calc} % calculations within tikz
\usetkzobj{all} % needed for tkz-euclide package
 
%%%%% MATH NOTATION %%%%%%%%

\newcommand\tn{\textnormal}

% systems
\newcommand{\R}{\mathbb{R}}
\newcommand{\C}{\mathbb{C}}
\newcommand{\Q}{\mathbb{Q}}
\newcommand{\N}{\mathbb{N}}
\newcommand{\Z}{\mathbb{Z}}

% notation tweaking
\renewcommand\phi\varphi  % normal \phi looks too much like the empty set.
\renewcommand\subset\subseteq 
\renewcommand\supset\supseteq  % to be careful about strict subsets and nonstrict subsets
\newcommand\st{:}

% divisibility
\newcommand\divides{\;|\;}
\newcommand\notdivides{\hspace*{-2pt}\not |\;}

% trig and geometry
\newcommand\degrees{^\circ}

%%%%%%%%%% THEOREMS AND RELATED STRUCTURES %%%%%
\renewcommand\emph[1]{\underline{\bf{#1}}} % terminology

\newtheorem{theorem}{Theorem}[section]
\newtheorem{proposition}[theorem]{Proposition}
\newtheorem{lemma}[theorem]{Lemma}
\newtheorem{corollary}[theorem]{Corollary}
\newtheorem{claim}{Claim}

\theoremstyle{definition}
\newtheorem{definition}[theorem]{Definition}
\newtheorem{example}[theorem]{Example}
\newtheorem{problem}[theorem]{Problem}
\newtheorem{conjecture}[theorem]{Conjecture}
\newtheorem{question}[theorem]{Question}
\newtheorem{remark}[theorem]{Remark}
\newtheorem{case}{Case}

\newtheorem*{theorem*}{Theorem}
\newtheorem*{example*}{Example}
\newtheorem*{question*}{Question}
\newtheorem*{claim*}{Claim}
\newtheorem*{definition*}{Definition}

\newenvironment{solution}{{\it Solution.} }{\hfill {\color{lightgray}$\blacksquare$}}

\newcommand\qedpart[1]{ \hfill \framebox(6,6){\tiny #1}}
\renewcommand\qed{\hfill \framebox(6,6){}}
%%%%%%%%%%%%%%%%%%%%%%%%%%%%%%%% 	
%%%%%%%% DOCUMENT BEGINS %%%%%%%%%%%%
%%%%%%%%%%%%%%%%%%%%%%%%%%%%%%%% 

\begin{document}

%%%%%% COVER PAGE %%%%%%%%%%%%%%% 
\pagenumbering{gobble} % no page number
\maketitle
\ifthenelse{\equal{\coursegraphic}{}} % insert course graphic if one exists
	{}
	{\begin{center}\includegraphics[width=3in]{\coursegraphic}\end{center}}
	
\vfill 
% copyleft
\begin{center} \includegraphics[width=1in]{by-nc-sa.png} \end{center}
\footnotesize{ This work is licensed under a Creative Commons Attribution-ShareAlike 3.0 Unported License. }

 % acknowledgments 
\footnotesize{
The Mathematics Of Doing, Understand, Learning, and Educating Secondary Schools (\MODULES) project is partially supported by funding from a collaborative grant of the National Science Foundation under Grant Nos.~DUE-1726707,1726804, 1726252, 1726723, 1726744, and 1726098.  Any opinions, findings, and conclusions or recommendations expressed in this material are those of the authors and do not necessarily reflect the views of the National Science Foundation.}
\newpage
%%%%%% TABLE OF CONTENTS %%%%%%%%%%%%%%% 	
\thispagestyle{plain} \pagenumbering{roman}  
\listoftodos
\tableofcontents
\newpage \pagenumbering{arabic}
%%%%%%%%%%%%%%%%%%%%%%%%%%%%%%%% 
%%%%%%%%%%%%%%%%%%%%%%%%%%%%%%%% 	
%%%%%% PART 1 %%%%%%%%%%%%%%%%%%%%%	
%%%%%%%%%%%%%%%%%%%%%%%%%%%%%%%% 
%%%%%%%%%%%%%%%%%%%%%%%%%%%%%%%% 	
\newpage 

\part{Introduction to Fields}

\section{Fields and Other Algebraic Structures}

%%Day 1

In this section we will begin our study of \emph{fields}. You've already encountered fields in your mathematical studies: the set of rational numbers $\mathbb{Q}$ 
and the set of real numbers $\mathbb{R}$ are fields, as is the set of complex numbers $\mathbb{C}$. The sets $\mathbb{Q}$, $\mathbb{R}$ and $\mathbb{C}$ are 
different in many ways, but here we will focus on the ways in which they are similar. 

We will also see that there are fields that are different from these three in some very important ways, but again, our goal for now is to understand the source
of their similarities. The number systems that we use on a daily basis are both very familiar and very strange. We can all recite facts about arithmetic
or the steps for solving a simple equation without much thought, but students have a knack for asking ``Why?'' at just the right time. The goal of this section
is to explore the foundations of the number systems we typically use for solving equations. This exploration will allow us to provide well grounded, thorough, and
pedagogically appropriate justifications for the steps we use in algebra every day to solve equations. But it will also allow us to explore exciting extensions of
our ordinary mathematical practices and allow us to connect equation solving to geometry in an intriguing way.

To begin, consider the equation
      \[ 3x + 8 = 14. \]
It's not hard to see that the solution to this equation is $x=2$: $3(2)+8 = 14$. Let's us solve this equation step-by-step, justifying each step along the 
way. First we will subtract $8$ from both sides:
\[ (3x+8)-8 = 14 -8.\]
(Note that we could also view this as adding $-8$ to both sides. The number $-8$ is known as the \emph{additive inverse} of $8$.) Applying the associative 
law on the left-hand side gives
\[ 3x + (8-8) = 6.\]
We know that $8-8=0$ so we have
\[ 3x + 0 = 6.\]
The number 0 is an \emph{additive identity}. That means adding 0 returns the value we added it to. So we have
\[ 3x = 6.\]
We now multiply each side by $1/3$ to obtain
\[ \frac{1}{3}(3x) = \frac{1}{3}\cdot 6. \]
Multiplication is associative, so we can write this as
\[ \left( \frac{1}{3}\cdot 3 \right)x = 2. \]
The number $1/3$ is the \emph{multiplicative inverse} of $3$, meaning that $\frac{1}{3}\cdot 3$ is equal to the \emph{multiplicative identity}; that is, $\frac{1}{3}\cdot 3 = 1$. Thus we have
\[ 1x = 2.\]
The number 1 is the \emph{multiplicative identity} meaning that $1x = x$. So we conclude that
\[ x = 2.\]

Let us analyze this situation more carefully. First note that the equation $3x+8 = 14$ uses two operations, called addition and multiplication. 
(Subtraction can always be defined in terms of addition, and division can be defined in terms of multiplication.)
We used some familiar properties of addition and multiplication such as associativity of addition and multiplication.

Above we multiplied by $1/3$ at point in the solution. Since $1/3$ is a rational number, we say that we solved this equation ``over the
rationals.'' But, notice that in this example we didn't really need to do this. Next we give a solution to the equation $3x+8=14$ ``over the
integers.'' We begin the same way:
\begin{align*}
  3x + 8 &= 14\\
 (3x+8)-8 &= 14 -8\\
 3x + (8-8) &= 6\\
 3x + 0 &= 6\\
 3x &= 6.
\end{align*}
Next we observe that $6 = 3(2)$ so we have
\[ 3x = 3(2).\]
One can prove that in the integers that if $a$, $b$, and $c$ are integers and $ab=ac$ then $b=c$. Using just that fact, we can conclude that
\[ x = 2.\]

\begin{task}
  \begin{itemize}
    \item Prove that if $a$, $b$, and $c$ are integers and $ab = ac$ then $b=c$. Remember - division is not allowed, we want to do this
      proof entirely in the integers.
    \item Can you solve the equation $3x+8=14$ over the natural numbers? (Here you're not allowed to use additive inverses!)
  \end{itemize}
\end{task}

The equation $3x+8=14$ can be solved over the rationals, integers, or natural numbers, but notice that the equation $3x+8=10$ cannot be solved over
the integers or natural numbers. The solution $x=2/3$ is a rational number and is not a natural number or integer. Notice that so long as $a$, $b$ and
$c$ are always rational numbers, $ax+b=c$ will always have a rational solution.  The same goes for equations with real or complex coefficients. On the
other hand, if $a$, $b$, and $c$ are integers, that does not guarantee that $ax+b=c$ will have an integer solution. We want to determine all of the 
properties necessary on a set of numbers for an equation such as $ax+b=c$ to always have a solution in that set. That is, we want to figure out
what makes a set of numbers like $\mathbb{Q}$, $\mathbb{R}$ and $\mathbb{C}$ in this regard. We will call such a set of numbers a field.

We begin with some terminology.

We call $0$ an additive identity because for any number $n$, $n+0 = 0 + n = n$. The number $0$ is also an additive identity in the set of complex numbers,
although more formally it is $0+0i$. A corresponding notion for multiplication exists - the multiplicative identity.

\begin{task}
  \begin{itemize}
    \item Consider the collection of all $2\times 2$ matrices whose entries are real numbers. Write down the additive identity of this set.
    \item How would you define the general notion of a \emph{multiplicative identity}? What is a multiplicative identity in $\mathbb{Q}$?
    \item Is there a multiplicative identity for the set of all $2\times 2$ matrices with real entries?
  \end{itemize}
\end{task}

Once we have a notion of an additive identity, we can define the notion of an additive inverse. We say that 
a number $b$ is an additive inverse of a number $a$ if and only if $a+b=b+a = 0$. 

\begin{task}
  How would you define the notion of a \emph{multiplicative inverse}? Give an example of a number $a$ and its 
  multiplicative inverse $b$.
\end{task}

A \emph{field} $\mathbb{F}$ is a collection of mathematical objects (possibly numbers, matrices, functions, etc.) with two operations, called
addition ($+$) and multiplication ($\cdot$), in which we can always solve an equation of the form
\[ ax + b = c\]
where $a,b,c\mathbb{F}$ and $a \neq 0$. The properties we need to make this happen are given in the following definition.

\begin{definition} A field $\mathbb{F}$ is a nonempty set together with two operations addition $+$ and multiplication $\cdot$ which satisfy the following
  properties, called the field axioms:
  \begin{enumerate}
    \item If $a,b\in \mathbb{F}$, there is a unique $a+b \in \mathbb{F}$.
    \item Addition is associative. That is, if $a,b,c\in\mathbb{F}$, then
      \[ (a+b)+c = a + (b+c).\]
    \item Addition is commutative. That is, if $a,b\in\mathbb{F}$, then
      \[ a+b = b+a.\]
    \item There is an additive identity in $\mathbb{F}$.
    \item If $a\in \mathbb{F}$, then $a$ has an additive inverse in $\mathbb{F}$.
    \item If $a,b\in\mathbb{F}$, then there is a unique $a\cdot b \in \mathbb{F}$.
    \item Multiplication is associative.
    \item Multiplication is commutative.
    \item There is a multiplicative identity in $\mathbb{F}$.
    \item If $a\in \mathbb{F}$ and $a\neq 0$, then there is a multiplicative inverse of $a$ in $\mathbb{F}$.
    \item Multiplication distributes over addition. That is, if $a,b,c\in\mathbb{F}$, then 
      \[ a\cdot (b+c) = a\cdot b + a\cdot c.\]
    \item The additive identity does not equal the multiplicative identity.
  \end{enumerate}
\end{definition}

Of course, you have seen fields before: the rational numbers $\mathbb{Q}$ and the real numbers $\mathbb{R}$ are both fields under their usual operations of
addition and multiplication. In fact, $\mathbb{Q}$ is a \emph{subfield} of $\mathbb{R}$.

\subsection{More on Identities and Inverses}

We all know that in the rational numbers there is only one additive identity: the number 0. But could it be that there is a field with more than one
additive identity? We have the following proposition:

\begin{proposition}
  In any field $\mathbb{F}$, then additive identity is unique.
\end{proposition}
\begin{proof}
  Suppose that we have additive identities $0$ and $z$ in $\mathbb{F}$. Since $0$ is an additive identity, we know that
  \[ 0 + z = z.\]
  But since $z$ is also an additive identity, we also know that
  \[ 0 + z = 0.\]
  So, we have that
  \[ z = 0 + z = 0.\]
  This proves that the additive identity in any field is unique.
\end{proof}

There are a couple of observations to make about this proof. First, a good general strategy for proving that something is unique is to assume that there are
two of them and then prove that they are equal.  If needed, you can also assume that your two proposed objects are not equal and derive a contradiction, but
notice that we did not need to do that in the proof above. Second, observe that besides using the definition of an additive identity, the only other property we
used to prove the proposition above is that addition is commutative.

Since the additive identity in any field is unique, we will always use the usual symbol $0$ to represent it, unless we have a good reason not to. 

\begin{task}
  Use the proof above as a model to show that in any field the multiplicative identity is unique.
\end{task}

Similarly, since the multiplicative identity in any field is unique, we will almost always use the usual symbol $1$ to represent it. 

There is a similar fact to observe with respect to additive and multiplicative inverses. For example, there is only one rational number whose sum with
$-\frac{1}{2}$ is 0, namely $\frac{1}{2}$. Similarly, there is only one rational number whose product with $-\frac{1}{2}$ is $1$, namely $-2$. 
Above you may have noticed that we said ``an additive inverse'' instead of ``the additive inverse,'' and
``a multiplicative inverse'' instead of ``the multiplicative inverse.'' We didn't want to suggest that they are unique, and were hoping that a reader might notice
our strange locution and question it. But now we are at a point where we are happy to admit that additive and multiplicative inverses are, in a sense, unique:

\begin{proposition}
  If $\mathbb{F}$ is a field and $a\in \mathbb{F}$, then its additive inverse is unique to it.
\end{proposition}
\begin{proof}
  Suppose that $\mathbb{F}$ is a field and that $a\in\mathbb{F}$. We want to prove that there is only one element $b\in\mathbb{F}$ so that
  \[ a+b=b+a=0.\]
  To this end, suppose that there are two such elements $b,c\in\mathbb{F}$. Then we have both:
  \begin{align*}
    a + b = b + a &= 0\\
    a + c = c + a &= 0
  \end{align*}
  Consider the sum $b + a + c$. On one hand we have
  \[ b + a + c = (b+a) + c = 0 + c = c.\]
  On the other hand we have
  \[ b + a + c = b+ (a + c) = b + 0 = b.\]
  Thus we conclude that $c=b$ and that every element in a field has a unique additive inverse.
\end{proof}

If $b$ is the additive inverse of $a$ we write $b = -a$. Note that $-a$ may be positive or negative. For example,
the additive inverse of $4$ is $-4$, but the additive inverse of $-5$ is $5$. This brings up an important point. 
When people see ``$-a$'' it is common to read it as ``minus $a$,'' or ``negative $a$.'' The least common thing for
people to say is ``the additive inverse of $a$.'' But, that's what we want you to do because it really helps to 
keep things straight as, for example, in the following proposition.

\begin{proposition}
  Suppose $a,b\in\mathbb{F}$. Then
  \begin{enumerate}
    \item[(a)] $-(-a) = a$
    \item[(b)] $-a = (-1)a$
    \item[(c)] $-(a+b)=(-a) + (-b)$
    \item[(d)] $-(a\cdot b) = (-a)\cdot b = a\cdot (-b)$.
  \end{enumerate}\label{prop: additive inverses}
\end{proposition}
\begin{proof}
The proof of item (a) is really an exercise in understanding the definition of the additive inverse. The expression $-a$ means ``the additive inverse
of $a$.'' So the expression ``$-(-a)$ means the additive inverse of $-a$. What is the additive inverse of $-a$? It's $a$ of course. That's because
\[ a + (-a) = (-a) + a = 0.\]
To prove (b), we want to so that $(-1)a$ is the additive inverse of $a$. How would we do that? Well, we must show that
\[ a + (-1)a = 0.\]
Here it goes:
\begin{align*}
  a + (-1)a &= 1a + (-1)a\\
            &= (1 + (-1))a\\
            &= 0\cdot a\\
            &= 0
\end{align*}
Thus $a+(-1)a =0$. Since addition is commutative we know that $a+(-1)a=(-1)a+a=0$. So $(-1)a$ fits the definition of an additive inverse of $a$. Since 
additive inverses are unique we conclude that $(-1)a=-a$. The proofs of parts (c) and (d) are left as exercises.
\end{proof}

It's a good idea to translate the statements in Proposition \ref{prop: additive inverses} into statements in ordinary language:
\begin{center}
  \begin{tabular}[]{ll}
    \textbf{Mathematical Statement} & \textbf{English Statement} \\
   $-(-a)=a$ & The additive inverse of the additive inverse of $a$ is $a$ itself.\\
   $-a = (-1)a$ & The additive inverse of $a$ is $-1$ times $a$.\\
   $-(a+b)=(-a)+(-b)$ & The additive inverse of a sum is the sum of the additive inverses.\\
   $-(a\cdot b) = (-a)\cdot b$  & The additive inverse of $a$ times $b$ is $b$ times the additive inverse of $a$. 
  \end{tabular}
\end{center}

Notice that in the proof above we had a nice, if slightly tricky, application of the distributive property. That trick is really helpful. Here's another 
application of it.

\begin{proposition}
  If $a\in\mathbb{F}$, then $a\cdot 0 = 0 \cdot a = 0$.
\end{proposition}
\begin{proof}
  We have 
  \begin{align*}
  a &= a\cdot 1\\
    &= a\cdot (1+0)\\
    &= a\cdot 1 + a\cdot 0\\
    &= a + a\cdot 0.
  \end{align*}
  Thus, $a = a + a \cdot 0$. Now add $-a$ to both sides:
  \begin{align*}
   (-a) + a &= (-a) + (a + a\cdot 0)\\
   0 &= (-a + a) + a\cdot 0\\
   0 &= 0 + a\cdot 0\\
   0 &= a\cdot 0
  \end{align*}
\end{proof}

\begin{task}
  In the proof above we only used field axioms, but did not identify which ones we used as we went along. For each step in the proof above, identify the
  field axioms that justify the step.
\end{task}

Now let's discuss multiplicative inverses. The fundamental facts about multiplicative inverses largely parallel the fundamental facts about multplicative inverses. In a field, there is a unique
mulplicative identity. We'll always call it $1$ unless we have a good reason not to. And, in a field every element except the additive identity has a multplicative inverse which is unique to it.
We will denote the multiplicative inverse of $a$ as $a^{-1}$. Because our experience with fields is mostly limited to $\mathbb{Q}$ and $\mathbb{R}$, it is common to reflexively think that 
\[ a^{-1} = \frac{1}{a}. \]
For example, $2^{-1} = 1/2$. And it is true that in $\mathbb{Q}$ and $\mathbb{R}$ (and even in $\mathbb{C}$), $a^{-1} = 1/a$ for nonzero $a$. However, as we will 
see in the example below, it is not true in every field that $a = 1/a$. 

\begin{example}
  We now define a field called $\mathbb{Z}_5$. We will avoid the complexities of a formal definition of $\mathbb{Z}_5$ and simply assert that 
  $\mathbb{Z}_5 = \{ 0, 1, 2, 3, 4\}$. The operations of addition ($+$) and multiplication ($\cdot$) are defined ``modulo 5.'' That means we find
  the regular sum or product and then divide by 5 and take the remainder as our answer. For example, in $\mathbb{Z}_5$:
  \begin{align*}
    3 + 4 &= 7\\
    7 \div 5 &= 1 \text{ R } 2.
  \end{align*}
  Since the sum of 3 and 4 is 7 and the remainder when we divide 7 by 5 is 2, we conclude that in $\mathbb{Z}_5$, $3 + 4 = 2$. Using that procedure, fill in
  the following addition table:

  \begin{center}
    \begin{tabular}{|c|c|c|c|c|c|}\\ \hline
      $+$ & 0 & 1 & 2 & 3 & 4 \\ \hline
      0   &   &   &   &   &   \\ \hline
      1   &   &   &   &   &   \\ \hline
      2   &   &   &   &   &   \\ \hline
      3   &   &   &   &   & 2 \\ \hline
      4   &   &   &   &   &   \\ \hline
    \end{tabular}
  \end{center}

  \begin{task}
    \begin{itemize}
      \item Find a number $a \in \mathbb{Z}_5$ with the property that $a + 2 = 2 + a = 0$. The number $a$ that you find is the additive inverse of $2$ in 
    $\mathbb{Z}_5$. That is, the value of $-2$ in $\mathbb{Z}_5$.
    \item Find the value of $-1$, $-3$, and $-4$ in $\mathbb{Z}_5$.
    \end{itemize}
  \end{task}

  We follow the analogous process for multiplication. For example, $4 \cdot 4 = 16$. When we divide $16$ by $5$ we get $3$ with a remainder of $1$. So we
  conclude that in $\mathbb{Z}_5$, $4 \cdot 4 = 1$. Using that procedure, fill in
  the following multiplication table:

  \begin{center}
    \begin{tabular}{|c|c|c|c|c|c|}\\ \hline
      $\cdot$ & 0 & 1 & 2 & 3 & 4 \\ \hline
      0   &   &   &   &   &   \\ \hline
      1   &   &   &   &   &   \\ \hline
      2   &   &   &   &   &   \\ \hline
      3   &   &   &   &   &   \\ \hline
      4   &   &   &   &   & 1 \\ \hline
    \end{tabular}
  \end{center}

  Since $4\cdot 4 = 1$ in $\mathbb{Z}_5$, we conclude that $4$ is its own multiplicative inverse in $\mathbb{Z}_5$. That is, $4^{-1}=4$ in $\mathbb{Z}_5$!

  \begin{task}
    \begin{itemize}
      \item Find the value of $2^{-1}$ and $3^{-3}$ in $\mathbb{Z}_5$.
    \end{itemize}
  \end{task}

\end{example}

\begin{task}
  \begin{itemize}
    \item Write down an addition table and multiplication table for $\mathbb{Z}_7 = \{0, 1, 2, 3, 4, 5, 6 \}$. For each element of $\mathbb{Z}_7$ find
      its additive inverse. Also find the multiplicative inverse of each nonzero element.
  \end{itemize}
\end{task}

The main point of the previous examples is that $-a$ does not always mean ``the negative of $a$,'' and $a^{-1}$ does not always mean $1/a$. In every context
it's safe to read $-a$ as ``the additive inverse of $a$'' and to read $a^{-1}$ as ``the multiplicative inverse of $a$.''

We have the following facts about multiplicative inverses. The proof of this proposition is left as an exercise.

\begin{proposition} Suppose that $\mathbb{F}$ is a field and that $a,b\in\mathbb{F}$ and $a,b\neq 0$.
  \begin{itemize}
    \item[(a)] $(a^{-1})^{-1} = a$
    \item[(b)] $(a\cdot b)^{-1}=b^{-1}\cdot a^{-1}$
    \item[(c)] $(-a)^{-1} = -a^{-1}$
  \end{itemize}\label{prop: multiplicative inverses}
\end{proposition}

\begin{task}
Again, it's helpful here to state the parts of this proposition in ordinary language. For example, part (a) asserts that ``the multiplicative inverse of
the multiplicative inverse of $a$ is $a$ itself.'' Do the same for parts (b) and (c).
\end{task}

\subsection{Field Extensions}

TODO

\subsection{Ordered Fields}

Above we gave the axioms that define a field. We add to those the following \emph{order axioms} to create an \emph{ordered field}.

\begin{definition} Suppose that $\mathbb{F}$ is a field. Then $\mathbb{F}$ is an \emph{ordered field} if there is an order
  relation $<$ on $\mathbb{F}$ that satisfies the following properties for any $a,b,c\in\mathbb{F}$:
\begin{itemize}
  \item (Trichotomy) Exactly one of the following holds: $a < b$, $a=b$ or $b < a$.
  \item (Transitivity) $a< b$ and $b < c$ implies that $a < c$.
  \item (Addition) If $a< b$, then $a+b < a+c$.
  \item (Multiplication) If $a < b$, $0 < c$, then $ac < bc$.
\end{itemize}
\end{definition}


\begin{definition}
  Suppose that $\mathbb{F}$ is an ordered field. An element $a\in\mathbb{F}$ is positive if $0 < a$ and $a$ is negative if $a < 0$.
\end{definition}

\begin{proposition}
  Suppose that $\mathbb{F}$ is an ordered field. Then
  \begin{enumerate}
    \item[(a)] $x\in \mathbb{F}$ is positive if and only if $-x$ is negative.
    \item[(b)] If $x,y \in \mathbb{F}$, then $x+y$ and $xy$ are positive.
    \item[(c)] If $x\neq 0$, then $x^2$ is positive.
    \item[(d)] $1$ is positive.
  \end{enumerate}
\end{proposition}
\begin{proof}
  For part (a) we proceed as follows. Suppose that $x\in\mathbb{F}$ is positive. By definition that means $0 < x$. Since $x\in\mathbb{F}$ it's additive
  inverse $-x$ is also in $\mathbb{F}$. By the addition axiom for ordering, since $0 < x$ we have
  \[ -x + 0 < -x + x.\]
  We know that $-x+0 = -x$ and that $-x + x = 0$. So we have $-x < 0$. Thus, if $x$ is positive, then $-x$ is negative.
\end{proof}

\begin{corollary}
  If $x$ and $y$ are positive, then $\frac{x}{y}$ and $\frac{y}{x}$ are positive.
\end{corollary}

\begin{corollary}
  $\mathbb{C}$ is not an ordered field.
\end{corollary}

\begin{proposition}
  In an ordered field, if $x$ is positive then $nx$ is positive for all $n$.
\end{proposition}

TODO: Somewhere above introduce integral elements.

\begin{proposition}
  In an ordered field, if $x$ is positive then $1/x$ is positive.
\end{proposition}

\begin{proposition} In an ordered field
  \begin{enumerate}
    \item[(a)] the product of a positive element and a negative element is negative.
    \item[(b)] the product of a negative element and a negative element is positive.
  \end{enumerate}
\end{proposition}


\begin{proposition}
  Suppose that $\mathbb{F}$ is an ordered field. Then for all $a,b\in\mathbb{F}$
  \begin{itemize}
    \item[(a)] If $a<b$ and $c>0$, then $ac < bc$.
    \item[(b)] If $a< b$ and $c < 0$, then $ac > bc$.
  \end{itemize}
\end{proposition}


\subsection{Other Fields}

\subsubsection{$\mathbb{Q}(\sqrt{n})$}

\subsubsection{$\mathbb{Q}[x]$}

\subsubsection{Finite Fields}

\subsection{The Archimedean Property and Completeness}

\subsubsection{What are the real numbers, really?}

\part{Constructible Numbers}

\subsection{Constructible Lengths}

\subsection{Quadratic Extensions}


\part{Three Famous Problems}

\subsection{Doubling the Cube}

\subsection{Trisecting an Angle}

\subsection{Squaring a Circle}


\end{document}



